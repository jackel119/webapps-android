\documentclass[a4paper]{article}

%% Language and font encodings
\usepackage[english]{babel}
\usepackage{listings}

%%useful package
\usepackage{amsmath}
\everymath{\displaystyle}

\usepackage{enumerate}
\usepackage[colorlinks=true, allcolors=blue]{hyperref}
\usepackage{graphicx}
\usepackage{float}
\usepackage{amssymb}

\usepackage{adjustbox} % Used to constrain images to a maximum size 


%%page size and margin
\usepackage[a4paper,top=3cm,bottom=2cm,left=3cm,right=3cm,marginparwidth=1.75cm]{geometry}

\title{C273: An Introduction to Law for Computer Scientists Coursework}
\author{Lan Yi (ly2715), Pobpawat Pordi(pp2916), Xiaokage Ying(xy1716), Zicong Ma(zm1216)}

\begin{document}
\maketitle
\section*{Q 1}
\begin{enumerate}
\item Licence
\\??????????????
\item Definition of Software
\\ This term does not violates any of Free Software Foundation's four freedoms.
\item Permitted Use
\\This term restricts user purpose to scientific research and education only, violating freedom 0 for which users should be free to run the program as they wish and for any purpose.
\\
\\In addition, this term prohibits users from distributing copies of the 'Software', either with or without modification, for commercial purposes. It violates freedom 2 and 3, which states that users are free to redistribute copies, either with or without modifications for any purpose, including but not limited to commercial purpose. 

\item Copyright and Distribution
\\Under this term, users are disallowed to sell 'software' or any part of it, or to distribute the "Software" in its original or modified form. It contradicts freedom 2 and 3 which affirm users' freedom to distribute and sell the "Software" with or without modification for any purpose and to anyone anywhere. 
\\
\\ The term also states that under permitted use, users may distribute modification to the "Software" only if they adhere some rules concerning modification. It first counters freedom 1 which asserts users' freedom to modify the "Software" however they want, and then contradicts freedom 3 which asserts users' freedom to distribute modified "Software" to anyone anywhere under no constrains. 
\\ 
\\ Moreover, the term prohibits users to sell its modification to this "Software", violating freedom 3 which states users should be free to sell its modification to anyone. 

\item Modifications
\\This term prevents modification of "Software" from constituting a working model on their own, violating freedom 1 which states that user should be free to modify the "Software" however they want.
\\
\\The term also requires users to attach prominent notices to the files that are modified and to the result that the modified "Software" generates. However, under freedom 1, users should be free to modify any part of "Software" and use the modification without mentioning that the modification exists, hence there is a conflict. 
\\
\\The term also states that "Error fixes and 'Software' modifications must be communicated to the coordinator of MPI-M model development/ model integration" and the users are required "to submit such modified parts of the 'Software' to MPI-M". This is another conflict to freedom 1 as the user should be free to choose whether he/she wants to communicate with the coordinator of MPI-M and whether he/she wants to submit the modification. 

\item Acknowledgements
\\????????????
\item Termination
\\This term terminates user's rights from using the "Software" if he/she does not comply the licence agreement. It breaches freedom 0, where users should be free to run "Software" whenever they want and should not be forbidden or stopped from making it run.
\item Fees
\\ This term does not violates any of Free Software Foundation's four freedoms.
\item Proprietary Rights
\\ This term does not violates any of Free Software Foundation's four freedoms.
\item Disclaimer of Warranty
\\ This term does not violates any of Free Software Foundation's four freedoms.
\item Limitation of Liability
\\ This term does not violates any of Free Software Foundation's four freedoms.
\item Support/Modifications
\\ This term does not violates any of Free Software Foundation's four freedoms.
\item Controlling Law and Severability
\\ This term does not violates any of Free Software Foundation's four freedoms.
\end{enumerate}


\section*{Q 2}


\end{document}